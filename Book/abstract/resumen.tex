%%%%%%%%%%%%%%%%%%%%%%%%%%%%%%%%%%%%%%%%%%%%%%%%%%%%%%%%%%%%%%%%%%%%%%%%%%%
%
% Generic template for TFC/TFM/TFG/Tesis
%
% $Id: resumen.tex,v 1.9 2015/06/05 00:10:31 macias Exp $
%
% By:
%  + Javier Macías-Guarasa. 
%    Departamento de Electrónica
%    Universidad de Alcalá
%  + Roberto Barra-Chicote. 
%    Departamento de Ingeniería Electrónica
%    Universidad Politécnica de Madrid   
% 
% Based on original sources by Roberto Barra, Manuel Ocaña, Jesús Nuevo,
% Pedro Revenga, Fernando Herránz and Noelia Hernández. Thanks a lot to
% all of them, and to the many anonymous contributors found (thanks to
% google) that provided help in setting all this up.
%
% See also the additionalContributors.txt file to check the name of
% additional contributors to this work.
%
% If you think you can add pieces of relevant/useful examples,
% improvements, please contact us at (macias@depeca.uah.es)
%
% You can freely use this template and please contribute with
% comments or suggestions!!!
%
%%%%%%%%%%%%%%%%%%%%%%%%%%%%%%%%%%%%%%%%%%%%%%%%%%%%%%%%%%%%%%%%%%%%%%%%%%%

\chapter*{Resumen}
\label{cha:resumen}
\markboth{Resumen}{Resumen}

\addcontentsline{toc}{chapter}{Resumen}

Este trabajo presenta el diseño e implementación de una plataforma de emulación para sistemas embebidos basada en QEMU, orientada a replicar una arquitectura compatible con el procesador \texttt{NOEL-V} de Gaisler, una implementación RISC-V de propósito general. Se desarrolló una máquina virtual funcional denominada \texttt{noel-srg}, que emula periféricos básicos como UART, GPIO y temporizadores, permitiendo ejecutar software embebido real compilado con la toolchain oficial.

Además del desarrollo del modelo, se diseñaron pruebas funcionales automatizadas integradas en un pipeline de integración continua utilizando GitHub Actions. Este enfoque permite validar el entorno emulado sin hardware físico, facilitando la detección de errores y asegurando la estabilidad del sistema a lo largo del tiempo.

El proyecto demuestra la viabilidad del uso de emulación como herramienta educativa y profesional en el ámbito embebido, y destaca el valor del software libre como base para desarrollos flexibles, reproducibles y colaborativos. Se plantea también una serie de mejoras y líneas futuras, como la ampliación del conjunto de periféricos, el soporte para arquitecturas multinúcleo y la eventual integración del código desarrollado al repositorio oficial de QEMU.


\textbf{Palabras clave:} \myThesisKeywords.

%%% Local Variables:
%%% TeX-master: "../book"
%%% End:


