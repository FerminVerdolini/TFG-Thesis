%%%%%%%%%%%%%%%%%%%%%%%%%%%%%%%%%%%%%%%%%%%%%%%%%%%%%%%%%%%%%%%%%%%%%%%%%%%
%
% Generic template for TFC/TFM/TFG/Tesis
%
% $Id: abstract.tex,v 1.9 2015/06/05 00:10:31 macias Exp $
%
% By:
%  + Javier Macías-Guarasa. 
%    Departamento de Electrónica
%    Universidad de Alcalá
%  + Roberto Barra-Chicote. 
%    Departamento de Ingeniería Electrónica
%    Universidad Politécnica de Madrid   
% 
% Based on original sources by Roberto Barra, Manuel Ocaña, Jesús Nuevo,
% Pedro Revenga, Fernando Herránz and Noelia Hernández. Thanks a lot to
% all of them, and to the many anonymous contributors found (thanks to
% google) that provided help in setting all this up.
%
% See also the additionalContributors.txt file to check the name of
% additional contributors to this work.
%
% If you think you can add pieces of relevant/useful examples,
% improvements, please contact us at (macias@depeca.uah.es)
%
% You can freely use this template and please contribute with
% comments or suggestions!!!
%
%%%%%%%%%%%%%%%%%%%%%%%%%%%%%%%%%%%%%%%%%%%%%%%%%%%%%%%%%%%%%%%%%%%%%%%%%%%

\chapter*{Abstract}
\label{cha:abstract}

\addcontentsline{toc}{chapter}{Abstract}

This project presents the design and implementation of an emulation platform for embedded systems based on QEMU, targeting a system architecture compatible with the \texttt{NOEL-V} processor from Gaisler, a general-purpose RISC-V implementation. A functional virtual machine, named \texttt{noel-srg}, was developed to emulate essential peripherals such as UART, GPIO, and timers, enabling the execution of real embedded software compiled with the official toolchain.

In addition to developing the model, automated functional tests were created and integrated into a continuous integration pipeline using GitHub Actions. This setup allows validating the emulated environment without physical hardware, helping detect errors and ensure system stability over time.

The project demonstrates the feasibility of emulation as both an educational and professional tool in embedded development, and highlights the value of free software as a foundation for flexible, reproducible, and collaborative solutions. Future work includes expanding peripheral support, enabling multicore simulation, and preparing the code for potential contribution to the official QEMU repository.


\textbf{Keywords:} \myThesisKeywordsEnglish.

%%% Local Variables:
%%% TeX-master: "../book"
%%% End:


