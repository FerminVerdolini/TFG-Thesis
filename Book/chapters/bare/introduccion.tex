%%%%%%%%%%%%%%%%%%%%%%%%%%%%%%%%%%%%%%%%%%%%%%%%%%%%%%%%%%%%%%%%%%%%%%%%%%%
%
% Generic template for TFC/TFM/TFG/Tesis
%
% $Id: introduccion.tex,v 1.1 2015/06/05 00:16:17 macias Exp $
%
% By:
%  + Javier Macías-Guarasa. 
%    Departamento de Electrónica
%    Universidad de Alcalá
%  + Roberto Barra-Chicote. 
%    Departamento de Ingeniería Electrónica
%    Universidad Politécnica de Madrid   
% 
% Based on original sources by Roberto Barra, Manuel Ocaña, Jesús Nuevo,
% Pedro Revenga, Fernando Herránz and Noelia Hernández. Thanks a lot to
% all of them, and to the many anonymous contributors found (thanks to
% google) that provided help in setting all this up.
%
% See also the additionalContributors.txt file to check the name of
% additional contributors to this work.
%
% If you think you can add pieces of relevant/useful examples,
% improvements, please contact us at (macias@depeca.uah.es)
%
% You can freely use this template and please contribute with
% comments or suggestions!!!
%
%%%%%%%%%%%%%%%%%%%%%%%%%%%%%%%%%%%%%%%%%%%%%%%%%%%%%%%%%%%%%%%%%%%%%%%%%%%

\chapter{Introducción}
\label{cha:introduction}

\begin{FraseCelebre}
  \begin{Frase}
    Desocupado lector, sin juramento me podrás creer que quisiera que este
    libro [...] fuera el más hermoso, el más gallardo y más discreto que
    pudiera imaginarse\footnote{Tomado de ejemplos del proyecto \texis{}.}.
  \end{Frase}
  \begin{Fuente}
    Miguel de Cervantes, Don Quijote de la Mancha
  \end{Fuente}
\end{FraseCelebre}


\section{Presentación}
\label{sec:intro-presentacion}

Blah, blah, blah.


\section{Organización de la memoria}
\label{sec:intro-organizacion-memoria}

Esta memoria se organiza en \ldots grandes capítulos. El primero \ldots


%%% Local Variables:
%%% TeX-master: "../book"
%%% End:
