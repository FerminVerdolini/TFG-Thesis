%%%%%%%%%%%%%%%%%%%%%%%%%%%%%%%%%%%%%%%%%%%%%%%%%%%%%%%%%%%%%%%%%%%%%%%%%%% 
% 
% Generic template for TFC/TFM/TFG/Tesis
% 
% By:
%  + Javier Macías-Guarasa.
%    Departamento de Electrónica
%    Universidad de Alcalá
%  + Roberto Barra-Chicote.
%    Departamento de Ingeniería Electrónica
%    Universidad Politécnica de Madrid
% 
% By: + Javier Macías-Guarasa. Departamento de Electrónica Universidad de Alcalá + Roberto Barra-Chicote. Departamento de Ingeniería Electrónica Universidad Politécnica de Madrid
% 
% Based on original sources by Roberto Barra, Manuel Ocaña, Jesús Nuevo, Pedro Revenga, Fernando Herránz and Noelia Hernández. Thanks a lot to all of them, and to the many anonymous contributors found (thanks to google) that provided help in setting all this up.
% 
% See also the additionalContributors.txt file to check the name of additional contributors to this work.
% 
% If you think you can add pieces of relevant/useful examples, improvements, please contact us at (macias@depeca.uah.es)
% 
% You can freely use this template and please contribute with comments or suggestions!!!
% 
%%%%%%%%%%%%%%%%%%%%%%%%%%%%%%%%%%%%%%%%%%%%%%%%%%%%%%%%%%%%%%%%%%%%%%%%%%% 

\chapter{Introducción}
\label{cha:introduccion}

El desarrollo de sistemas embebidos modernos presenta crecientes desafíos vinculados con la validación temprana del software,
la integración con hardware personalizado y la necesidad de mantener ciclos de desarrollo ágiles y confiables. En este contexto,
el uso de herramientas de emulación y virtualización se ha vuelto fundamental para abordar estas problemáticas,
permitiendo simular entornos completos de ejecución sin necesidad de contar con el hardware físico desde las primeras etapas del desarrollo.

Este trabajo se enmarca en esa problemática, proponiendo la extensión de una plataforma de emulación basada en QEMU
para incorporar soporte funcional a una arquitectura de sistema embebido concreta: la máquina \texttt{NOEL-V} desarrollada por Gaisler
para aplicaciones aeroespaciales, se basa en el estándar abierto RISC-V y se integra con la biblioteca GRLIB, ampliamente utilizada en sistemas críticos.

A través del diseño, implementación y validación de nuevos periféricos dentro del modelo de máquina virtual,
este proyecto busca ofrecer un entorno reproducible y automatizado para la ejecución de pruebas funcionales sobre software embebido.
La solución se integra, además, dentro de un flujo de integración continua, lo que permite verificar de forma automática la estabilidad
de la plataforma ante cada cambio en el código.

En las siguientes secciones se detallan la motivación del trabajo, los objetivos planteados y el enfoque de desarrollo adoptado para su implementación.

\section{Motivación}

El desarrollo de sistemas embebidos conlleva múltiples desafíos técnicos, especialmente en las etapas iniciales donde 
el hardware aún no está disponible o es costoso de replicar. Esta limitación impacta tanto en el entorno profesional como en el académico.
La posibilidad de emular plataformas completas a nivel de sistema permite mitigar estos obstáculos, facilitando el desarrollo temprano,
la depuración intensiva y la validación funcional sin requerir acceso físico al hardware \cite{zabolotny2021qemu,qemu}.

En el contexto educativo, aprender sobre sistemas embebidos suele implicar el uso de placas específicas,
dispositivos periféricos y entornos de programación restrictivos. Esto genera barreras de entrada importantes para estudiantes 
y centros con recursos limitados. En este sentido, la emulación ofrece una vía accesible para explorar arquitecturas complejas, 
experimentar con software de bajo nivel y adquirir experiencia práctica sin los costos ni riesgos asociados al hardware real \cite{zabolotny2021qemu}.

En el ámbito industrial, y especialmente en empresas pequeñas o con procesos informales, el testing de software embebido
suele estar subestimado o directamente ausente. La necesidad de disponer de múltiples bancos de prueba físicos, 
junto con la dificultad para reproducir entornos consistentes, complica la implementación de flujos de validación automatizados. 
Esta situación lleva a una falta de control sobre la calidad del software, mayor riesgo de regresiones y una dependencia excesiva de pruebas manuales \cite{cinque2021qemu}.

El uso de plataformas de emulación como QEMU, combinadas con estrategias de integración continua y diseño cuidadoso de los tests, 
permite abordar estos problemas de manera eficaz. No solo se facilita el desarrollo incremental y reproducible del software embebido,
sino que se sientan las bases para prácticas más profesionales y escalables en el diseño de sistemas críticos \cite{githubactions}.


\section{Objetivos}

El presente trabajo de fin de grado tiene como objetivo general desarrollar un entorno virtualizado funcional y confiable que permita validar
software embebido sin requerir hardware físico. Para ello, se extiende el emulador QEMU con una nueva máquina virtual basada en la arquitectura 
\texttt{NOEL-V}, de Gaisler, replicando periféricos comunes como UART, GPIO y temporizadores incluidos en la biblioteca \texttt{GRLIB}.

Este entorno busca ser suficientemente flexible como para permitir la incorporación de nuevas plataformas o variantes de hardware
con un esfuerzo razonable. De esta forma, cualquier desarrollador podría adaptar fácilmente el modelo virtual a uno nuevo basado en GRLIB o similar,
acelerando las etapas de desarrollo y prueba.

Los objetivos específicos del proyecto son:

\begin{itemize}
	\item Analizar las características de QEMU como plataforma de emulación para sistemas embebidos, en particular para arquitecturas RISC-V.
	\item Estudiar el entorno de ejecución de la plataforma \texttt{NOEL-V} y su integración con el ecosistema GRLIB.
	\item Diseñar y desarrollar una máquina virtual personalizada en QEMU que represente una plataforma basada en \texttt{NOEL-V}.
	\item Emular periféricos básicos (UART, GPIO, timer) de forma funcional, garantizando la comunicación con software real.
	\item Diseñar y ejecutar pruebas automáticas sobre el entorno emulado, asegurando la validez del modelo virtual.
	\item Integrar un sistema de integración continua (CI) que automatice la ejecución de los tests ante cambios en el código.
	\item Documentar el proceso seguido para extender QEMU, facilitando su replicación o adaptación a nuevas plataformas similares.
\end{itemize}

Con este enfoque, se espera contribuir a la reducción de la dependencia del hardware físico durante las etapas tempranas del desarrollo embebido,
al tiempo que se habilita un marco reproducible, automatizable y de bajo costo para pruebas funcionales.


\section{Desarrollo general del sistema}

Este trabajo se centró en la extensión del emulador \texttt{QEMU} \cite{qemu} para incorporar una nueva 
plataforma basada en la arquitectura \texttt{RISC-V} \cite{waterman2011riscv}, modelando una 
plataforma virtual compatible con el ecosistema de la empresa \texttt{Gaisler}. 
La plataforma diseñada, denominada \texttt{noel-srg}, reproduce el comportamiento basado en el procesador 
\texttt{NOEL-V} \cite{gaisler-noelv} y periféricos típicos incluidos en la biblioteca \texttt{GRLIB} \cite{grlib}, 
ampliamente utilizada en sistemas embebidos críticos.

Uno de los pilares fundamentales de este desarrollo es el aprovechamiento de herramientas de código abierto. 
La elección de \texttt{QEMU} no solo se debe a su madurez técnica y su amplio soporte de arquitecturas 
\cite{cinque2021qemu}, sino también a su naturaleza \textit{open-source}, que permite acceder, comprender y modificar 
el código fuente del emulador. Esta característica resulta clave tanto en contextos educativos —donde promueve la 
experimentación y el aprendizaje profundo— como en entornos profesionales, donde la adaptabilidad es esencial para 
afrontar requisitos particulares sin depender de soluciones privativas \cite{zabolotny2021qemu}.

El proceso de desarrollo se llevó a cabo respetando las convenciones del proyecto oficial de \texttt{QEMU}, 
manteniendo la compatibilidad con sus herramientas y estructuras internas. Se definió un modelo de máquina que describe 
la topología de dispositivos, la asignación de memoria y el comportamiento funcional de periféricos como la UART, 
el temporizador y los GPIOs. Estos componentes fueron implementados y verificados mediante binarios de prueba escritos 
en lenguaje C, compilados con la toolchain específica de Gaisler \texttt{riscv-gaisler-elf-gcc}.

Las pruebas iniciales se realizaron en un entorno local basado en Linux, y posteriormente se implementó un sistema de 
integración continua utilizando \texttt{GitHub Actions} \cite{githubactions}. Este \textit{pipeline} permite ejecutar 
de manera automática los tests funcionales cada vez que se introduce un cambio en el repositorio, asegurando la estabilidad 
de la plataforma virtual y fomentando un flujo de trabajo colaborativo y profesional.

El modelo resultante fue concebido con una arquitectura modular y extensible, lo que permite su adaptación a nuevas 
variantes de plataformas o la incorporación de periféricos adicionales con relativa facilidad. Esto lo convierte en una 
base sólida tanto para proyectos de investigación como para iniciativas de formación en diseño de sistemas embebidos.


\subsection*{Estructura del documento}

El presente trabajo se organiza en los siguientes capítulos:

\begin{itemize}
	\item \textbf{Capítulo 2 – Estudio teórico y estado del arte:} Se presentan los conceptos fundamentales sobre emulación y virtualización 
     en sistemas embebidos. También se revisan casos reales del uso de QEMU en contextos académicos e industriales, y se analiza su relevancia 
     en entornos RISC-V y pipelines de integración continua.

	\item \textbf{Capítulo 3 – Emulador Basado en QEMU de RISCV-NOEL:} Se describe el proceso de implementación de una nueva máquina virtual en QEMU,
     basada en la arquitectura NOEL de Gaisler. Se detallan las decisiones de diseño, el soporte de periféricos y las herramientas utilizadas para 
     extender QEMU.

	\item \textbf{Capítulo 4 – Pruebas de validación funcional e integración continua:} Se documentan los tests funcionales desarrollados para validar
     la plataforma virtual (UART, GPIO, temporizador, interrupciones), junto con la automatización de pruebas mediante un entorno de integración 
     continua basado en GitHub Actions.

	\item \textbf{Capítulo 5 – Conclusiones y líneas futuros:} Se sintetizan los logros del proyecto, se reflexiona sobre los aportes obtenidos
     y se plantean posibles líneas de trabajo a futuro.
\end{itemize}
