%%%%%%%%%%%%%%%%%%%%%%%%%%%%%%%%%%%%%%%%%%%%%%%%%%%%%%%%%%%%%%%%%%%%%%%%%%%
%
% Generic template for TFC/TFM/TFG/Tesis
%
% $Id: acronymsgl.tex,v 1.8 2015/06/05 00:10:31 macias Exp $
%
% By:
%  + Javier Macías-Guarasa. 
%    Departamento de Electrónica
%    Universidad de Alcalá
%  + Roberto Barra-Chicote. 
%    Departamento de Ingeniería Electrónica
%    Universidad Politécnica de Madrid   
% 
% Based on original sources by Roberto Barra, Manuel Ocaña, Jesús Nuevo,
% Pedro Revenga, Fernando Herránz and Noelia Hernández. Thanks a lot to
% all of them, and to the many anonymous contributors found (thanks to
% google) that provided help in setting all this up.
%
% See also the additionalContributors.txt file to check the name of
% additional contributors to this work.
%
% If you think you can add pieces of relevant/useful examples,
% improvements, please contact us at (macias@depeca.uah.es)
%
% You can freely use this template and please contribute with
% comments or suggestions!!!
%
%%%%%%%%%%%%%%%%%%%%%%%%%%%%%%%%%%%%%%%%%%%%%%%%%%%%%%%%%%%%%%%%%%%%%%%%%%%

% You can change the way the entries appear the first time they are
% used. I've used italics by default. I found a problem if using this:
% LaTeX adds an extra space after the acronym, so I'm commenting it out
% (if you find a solution, please let me know)
%\defglsdisplayfirst[\acronymtype]{\textit{#1}} % EDIT this if required

% This may lead to problems... I don't know how to fix it in case the
% column for acronym is wider than 0.3\linewidth
% \setlength{\glsdescwidth}{0.7\linewidth}       % EDIT this if required

% % Set language specific definitions...
% \ifthenelse{\equal{\myLanguage}{english}}
% {
% \printglossary[type=\acronymtype,style=super,nonumberlist=true,title=List of Acronyms,toctitle=List of Acronyms]
% \addcontentsline{toc}{chapter}{List of Acronyms}
% }
% {
% \printglossary[type=\acronymtype,style=super,nonumberlist=true,title=Lista de acrónimos,toctitle=Lista de acrónimos]
% \addcontentsline{toc}{chapter}{Lista de acrónimos}
% }


%%% Local Variables:
%%% TeX-master: "../book"
%%% End:


