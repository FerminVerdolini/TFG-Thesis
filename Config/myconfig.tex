%%%%%%%%%%%%%%%%%%%%%%%%%%%%%%%%%%%%%%%%%%%%%%%%%%%%%%%%%%%%%%%%%%%%%%%%%%%
%
% Generic template for TFC/TFM/TFG/Tesis
%
% $Id: myconfig.tex,v 1.39 2020/03/24 17:33:24 macias Exp $
%
% By:
%  + Javier Macías-Guarasa. 
%    Departamento de Electrónica
%    Universidad de Alcalá
%  + Roberto Barra-Chicote. 
%    Departamento de Ingeniería Electrónica
%    Universidad Politécnica de Madrid   
% 
% Based on original sources by Roberto Barra, Manuel Ocaña, Jesús Nuevo,
% Pedro Revenga, Fernando Herránz and Noelia Hernández. Thanks a lot to
% all of them, and to the many anonymous contributors found (thanks to
% google) that provided help in setting all this up.
%
% See also the additionalContributors.txt file to check the name of
% additional contributors to this work.
%
% If you think you can add pieces of relevant/useful examples,
% improvements, please contact us at (macias@depeca.uah.es)
%
% You can freely use this template and please contribute with
% comments or suggestions!!!
%
%%%%%%%%%%%%%%%%%%%%%%%%%%%%%%%%%%%%%%%%%%%%%%%%%%%%%%%%%%%%%%%%%%%%%%%%%%%

%%%%%%%%%%%%%%%%%%%%%%%%%%%%%%%%%%%%%%%%%%%%%%%%%%%%%%%%%%%%%%%%%%%%%%%%%%% 
%
% Contents of this file:
% + Definition of variables controlling compilation flavours
% + Definition of your own commands (samples provided)
%
% You must edit it to suit to your specific case
%
% Specially important are the definition of your variables (title of the
% book, your degree, author name, email, advisors, keywords (in Spanish
% and English), year, ... They will be used in generating the adequate
% front and cover pages, etc. automagically...
%
%%%%%%%%%%%%%%%%%%%%%%%%%%%%%%%%%%%%%%%%%%%%%%%%%%%%%%%%%%%%%%%%%%%%%%%%%%% 

%%%%%%%%%%%%%%%%%%%%%%%%%%%%%%%%%%%%%%%%%%%%%%%%%%%%%%%%%%%%%%%%%%%%%%%%%%% 
% BEGIN Set my own variables (control compilation for different flavours)

% Control language specific modifications
% This can be english or spanish
\newcommand{\myLanguage}{spanish}

% Control compilation flavour (for PFCs, TFMs, TFGs, Thesis, etc...)
% Degree (titulación), can be:
% GITT   - Grado en Ingeniería en Tecnologías de la Telecomunicación
% GIEC   - Grado en Ingeniería Electrónica de Comunicaciones
% GIT    - Grado en Ingeniería Telemática
% GIST   - Grado en Ingeniería en Sistemas de Telecomunicación
% GIC    - Grado en Ingeniería de Computadores
% GII    - Grado en Ingeniería Informática
% GSI    - Grado en Sistemas de Información
% GISI   - Grado en Ingeniería en Sistemas de Información
% GIEAI  - Grado en Ingeniería en Electrónica y Automática Industrial
% GITI   - Grado en Ingeniería en Tecnologías Industriales
% MUSEA  - Máster Universitario en Sistemas Electrónicos Avanzados. Sistemas Inteligentes
% MUIT   - Máster Universitario en Ingeniería de Telecomunicación
% MUII   - Máster Universitario en Ingeniería Industrial
% MUIE   - Máster Universitario en Ingeniería Electrónica
% MUCTE  - Máster Universitario en Ciencia y Tecnología desde el Espacio
% PHDUAH - Doctorado UAH
% PHDUPM - Doctorado UPM
%
% GEINTRARR - Geintra Research Report (alpha support)
%
% And the already deprecated pre-Bologna degrees (still active for
% sentimental reasons :-)):
% IT     - Ingeniería de Telecomunicación
% IE     - Ingeniería Electrónica
% ITTSE  - Ingeniería Técnica de Telecomunicación, Sistemas Electrónicos
% ITTST  - Ingeniería Técnica de Telecomunicación, Sistemas de Telecomunicación
% ITI    - Ingeniería Técnica Industrial, Electrónica Industrial 
%
% You can include additional degrees and modify Config/myconfig.tex
% Config/postamble.tex and Book/cover/cover.tex, generating new specific
% cover files if needed. Contact me if you want additional details
\newcommand{\myDegree}{GIC}

\newcommand{\myFlagSplittedAdvisors}{true} % if false it will set
                                % "Tutores/Advisors" in the cover
                                % pages. Otherwise it will split in
                                % Tutor/Cotutor Advisor/Co-advisor


\newcommand{\mySpecialty}{} % New in TFGs from 20151218!

%%%%%%%%%%%%%%%%%%%%%%%%%%%%%%%%%%%%%%%%%%%%%%%%%%%%%%%%%%%%%%%%%%%%%%%%%%%
% General document information
\newcommand{\myBookTitleSpanish}{Emulador Basado en QEMU de RISCV-NOEL}
\newcommand{\myBookTitleEnglish}{Undefinded title}
\newcommand{\myThesisKeywords}{emulación, sistemas embebidos, QEMU, RISC-V, GRLIB, integración continua, software libre} % (máximo de cinco)
\newcommand{\myThesisKeywordsEnglish}{emulation, embedded systems, QEMU, RISC-V, GRLIB, continuous integration, open-source software} % (up to a maximum of five)
\newcommand{\myConfidentialContent}{No} % This can be Yes or No, used
                                % in MUII as of July 2021

%%%%%%%%%%%%%%%%%%%%%%%%%%%%%%%%%%%%%%%%%%%%%%%%%%%%%%%%%%%%%%%%%%%%%%%%%%%
% Author data
\newcommand{\myAuthorName}{Fermín}
\newcommand{\myAuthorSurname}{Verdolini}
\newcommand{\myAuthorFullName}{\myAuthorName{} \myAuthorSurname{}}
\newcommand{\myAuthorGender}{male} 
\newcommand{\myAuthorEmail}{fermin.verdolini@edu.uah.es}
\newcommand{\myAuthorDNI}{} 
% Personal details for the anteproyecto request
% Not required in some cases
\newcommand{\myAuthorStreet}{C/ Via Complutense, 71}
\newcommand{\myAuthorCity}{Alcalá de Henares}
\newcommand{\myAuthorPostalCode}{28805}
\newcommand{\myAuthorProvince}{Madrid}
\newcommand{\myAuthorTelephone}{602493556}


%%%%%%%%%%%%%%%%%%%%%%%%%%%%%%%%%%%%%%%%%%%%%%%%%%%%%%%%%%%%%%%%%%%%%%%%%%%
% Advisor data
\newcommand{\myAcademicTutorFullName}{Antonio Da Silva Fariña} 
\newcommand{\myAcademicTutorGender}{male}
\newcommand{\myAcademicTutorDNI}{}
\newcommand{\myAcademicTutorDepartmentOrInstitution}{} 

%%%%%%%%%%%%%%%%%%%%%%%%%%%%%%%%%%%%%%%%%%%%%%%%%%%%%%%%%%%%%%%%%%%%%%%%%%%
% CoAdvisor data
\newcommand{\myCoTutorFullName}{Miguel Solinas}
\newcommand{\myCoTutorGender}{male}
\newcommand{\myCoTutorDNI}{}
\newcommand{\myCoTutorDepartmentOrInstitution}{Universidad Nacional de Córdoba, Argentina} 

%%%%%%%%%%%%%%%%%%%%%%%%%%%%%%%%%%%%%%%%%%%%%%%%%%%%%%%%%%%%%%%%%%%%%%%%%%%
% Affiliation
\newcommand{\mySchool}{Escuela Politécnica Superior}
\newcommand{\myUniversity}{Universidad de Alcalá de Henares}
\newcommand{\myUniversityAcronym}{UAH}

\newcommand{\myDepartment}{Departamento de Ciencias de la Computación}
\newcommand{\myDepartmentEnglish}{}

\newcommand{\myPhDProgram}{}
\newcommand{\myPhDProgramEnglish}{}

\newcommand{\myResearchGroup}{}


%%%%%%%%%%%%%%%%%%%%%%%%%%%%%%%%%%%%%%%%%%%%%%%%%%%%%%%%%%%%%%%%%%%%%%%%%%%
% Tribunal members & department staff
\newcommand{\myTribunalPresident}{}
\newcommand{\myTribunalFirstSpokesperson}{}
\newcommand{\myTribunalSecondSpokesperson}{} 
\newcommand{\myTribunalAlternateMember}{}
\newcommand{\myTribunalSecretary}{}
\newcommand{\myDepartmentSecretary}{} % Por TFGs & TFMs & MUSEA-TFMs paperwork
\newcommand{\myDepartmentSecretaryGender}{}                 % Por TFGs & TFMs & MUSEA-TFMs paperwork
\newcommand{\myTFMComisionPresident}{}      % Por MUIE TFMs, no need to define gender... presidentE always

%%%%%%%%%%%%%%%%%%%%%%%%%%%%%%%%%%%%%%%%%%%%%%%%%%%%%%%%%%%%%%%%%%%%%%%%%%%
% Calendar dates 

\newcommand{\myThesisProposalDate}{17 de Octubre de 2024} % "Anteproyecto" date

\newcommand{\myThesisDepositDate}{1 de julio de 2025}
\newcommand{\myThesisDepositDateEnglish}{December 1\textsuperscript{st}, 2022}

% For RR, myThesisDefenseDate is date to be shown in the cover
\newcommand{\myThesisDefenseYear}{2023}
\newcommand{\myThesisDefenseDate}{16 de enero de \myThesisDefenseYear{}}
\newcommand{\myThesisdefenseDateEnglish}{January 16\textsuperscript{th}, \myThesisDefenseYear{}}
% If you prefer British English for the date, use this:
% \newcommand{\myThesisdefenseDateEnglish}{6\textsuperscript{th} of January, 2018}

\newcommand{\myPaperworkDate}{10 de enero de 2023}

%%%%%%%%%%%%%%%%%%%%%%%%%%%%%%%%%%%%%%%%%%%%%%%%%%%%%%%%%%%%%%%%%%%%%%%%%%%
% Open publication details
\newcommand{\myAuthorizationOpenPublishing}{Yes}
\newcommand{\myAuthorizationOpenPublishingEmbargoMonths}{0} % Can be
                                                          %  0,  6, 12,
                                                          % 18, 24 

%\newcommand{\myResearchVicerrector}{Excma. Sra. María Luisa Marina Alegre}
\newcommand{\myResearchVicerrector}{Excmo. Sr. Francisco J. de la Mata de la Mata}
\newcommand{\myResearchVicerrectorGender}{male}

% Copyright related issues
\newcommand{\myCopyrightStatement}{\myAuthorFullName{}. Some rights reserved. This document is under terms of Creative Commons license Attribution - Non Commercial - Non Derivatives.}
\newcommand{\myLicenseURL}{http://creativecommons.org/licenses/by-nc-nd/3.0/es/}

\newcommand{\myResearchReportID}{RR-2021-01}


%%%%%%%%%%%%%%%%%%%%%%%%%%%%%%%%%%%%%%%%%%%%%%%%%%%%%%%%%%%%%%%%%%%%%%%%%%%
% Link color definition
% Color links of the toc/lot/lof entries
%\newcommand{\mytoclinkcolor}{blue}
\newcommand{\mytoclinkcolor}{black}
%\newcommand{\myloflinkcolor}{red}
\newcommand{\myloflinkcolor}{black}
%\newcommand{\mylotlinkcolor}{green}
\newcommand{\mylotlinkcolor}{black}

% This is used in cover/extralistings.tex
%\newcommand{\myothertoclinkcolor}{magenta}
\newcommand{\myothertoclinkcolor}{black}

% Other color links in the document
\newcommand{\mylinkcolor}{blue}
%\newcommand{\mylinkcolor}{black}

% Color links to urls and cites
\newcommand{\myurlcolor}{blue}
%\newcommand{\myurlcolor}{black}
%\newcommand{\mycitecolor}{green}
\newcommand{\mycitecolor}{blue}
%\newcommand{\mycitecolor}{black}

% END Set my own variables (control compilation for different flavours)
%%%%%%%%%%%%%%%%%%%%%%%%%%%%%%%%%%%%%%%%%%%%%%%%%%%%%%%%%%%%%%%%%%%%%%%%%%% 

%%%%%%%%%%%%%%%%%%%%%%%%%%%%%%%%%%%%%%%%%%%%%%%%%%%%%%%%%%%%%%%%%%%%%%%%%%% 
% BEGIN My own commands section 
% Define your own commands here

% This one is to define a specific format for english text in a Spanish
% document
\DeclareRobustCommand{\texten}[1]{\textit{#1}}

\def\ci{\perp\!\!\!\perp}

% Various examples of commonly used commands
\newcommand{\circulo}{\large $\circ$}
\newcommand{\asterisco}{$\ast$}
\newcommand{\cuadrado}{\tiny $\square$}
\newcommand{\triangulo}{\scriptsize $\vartriangle$}
\newcommand{\triangv}{\scriptsize $\triangledown$}
\newcommand{\diamante}{\large $\diamond$}

\newcommand{\new}[1]{\textcolor{magenta}{#1 }}
\newcommand{\argmax}[1]{\underset{#1}{\operatorname{argmax}}}

% This is an example used in the sample chapters
\newcommand{\verticalSpacingSRPMaps}{-0.3cm}




%%%%%%%%%%%%%%%%%%%%%%%%%%%%%%%%%%%%%%%%%%%%%%%%%%%%%%%%%%%%%%%%%%%%%%%%%%%
% Deprecated and less useful definitions, just keep them...
\newcommand{\myFirstAdvisorFullName}{\myAcademicTutorFullName} % This is deprecated: set to academic tutor
\newcommand{\mySecondAdvisorFullName}{\myCoTutorFullName} % This is deprecated: set to cotutor
\newcommand{\myFirstAdvisorDNI}{\myAcademicTutorDNI} % Deprecated: set to that of academic tutor
\newcommand{\mySecondAdvisorDNI}{\myCoTutorDNI} % Deprecated set to that of cotutor
\newcommand{\mybookFigure}{alumno} % Deprecated, was required
                                % for TFG's: the type of adscription of
                                % the author signing the agreement
                                % (should be "alumno" in most cases)

\newcommand{\myUPMdegree}{Ingeniero de Telecomunicación} % Used in UPM

% END My own commands section 
%%%%%%%%%%%%%%%%%%%%%%%%%%%%%%%%%%%%%%%%%%%%%%%%%%%%%%%%%%%%%%%%%%%%%%%%%%% 

%%% Local Variables:
%%% TeX-master: "../book"
%%% End:


